\documentclass{beamer}
\usepackage{style}

\title{M05 - Mini project}
\subtitle{Human Activity Recognition from Continuous Ambient Sensor Data}
\author{Steve Devènes, Amara Spano}
\institute{Unidistance}
\date{\today}

\begin{document}
	
	\begin{frame}
		\titlepage
	\end{frame}
	
	\begin{frame}
		\frametitle{Outline}
		\tableofcontents
	\end{frame}

%	\begin{frame}
%		\frametitle{Selected project}
%	\end{frame}

	\begin{frame}
		\frametitle{Working hypothesis}
		It possible to perform human activity recognition from continuous ambient sensor data
	\end{frame}

	\begin{frame}
		\frametitle{Data}
		Data is available online on the UC Irvine machine learning repository.
		
		There are two evaluation protocols:
		\begin{figure}
			\centering
			%\includegraphics[width=\linewidth]{idiap_web.jpg}
			\missingfigure{Protocols illustration}
		\end{figure}
	
		To change the evaluation protocoles
	\end{frame}

	\begin{frame}
		\frametitle{Workflow}
		For different experiments:
		\begin{enumerate}
			\item Load the training data
			\item Create and train the model
			\item Load the test data
			\item Make prediction on test data
			\item Print the confusion matrix for the model evaluation
		\end{enumerate}
	
		The experiments run sequentially.
	\end{frame}

	\begin{frame}
		\frametitle{Version control}
		On github at \url{https://github.com/sdevenes/M05_MiniProject}
		
		The work is organized using github issues to create and assign tasks.
		
		The general approach:
		\begin{itemize}
			\item Create 1 branch per feature named \textit{feature/feature\_name}
			\item When the feature is complete, do a pull request with the other as a reviewer
		\end{itemize}
	\end{frame}

%	\begin{frame}
%		\frametitle{Version control}
%		\begin{figure}
%			\centering
%			%\includegraphics[width=\linewidth]{idiap_web.jpg}
%			\missingfigure{Git tree illustration}
%		\end{figure}
%	\end{frame}


%	\begin{frame}
%		\frametitle{Unit Testing and CI}
%	\end{frame}

	\begin{frame}
		\frametitle{Documentation}
		Each function is commented with a docstring.
		
		We plan to use the documentation generator Sphinx.
	\end{frame}

%	\begin{frame}
%		\frametitle{Packaging}
%	\end{frame}

%	\begin{frame}
%		\frametitle{Final remarks}
%	\end{frame}
\end{document}